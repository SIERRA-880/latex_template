\documentclass{article}
\usepackage{svg}
\usepackage{amsmath}
\usepackage[utf8]{inputenc}
\usepackage[french]{babel}
\usepackage{graphicx}
\usepackage[T1]{fontenc}
\usepackage{hyperref}
\hypersetup{
    colorlinks,
    citecolor=black,
    filecolor=black,
    linkcolor=black,
    urlcolor=black
}
\usepackage[obeyspaces,spaces]{url}
\usepackage{tikz}
\usetikzlibrary{shapes,positioning}
\graphicspath{images/}

\usepackage{blindtext}

\usepackage{subfiles} % Best loaded last in the preamble

\title{Rapport projet optimisation linéaire}
\author{ }
\date{ }

\begin{document}
\begin{titlepage}
    \begin{center}
        
        {\Large Université de Mons}\\[1ex]
        {\Large Faculté Polytechnique}\\[1ex]
        
        \newcommand{\HRule}{\rule{\linewidth}{0.3mm}}
        % Title
        \HRule \\[0.3cm]
        { \LARGE \bfseries Projet optimisation linéaire \\[0.3cm]}
        { \LARGE \bfseries Rapport \\[0.1cm]} % Commenter si pas besoin
        \HRule \\[1.5cm]
        
        % Author and supervisor
        \begin{minipage}[t]{0.45\textwidth}
            \begin{flushleft} \large
                \emph{Professeur:}\\
                Nicolas \textsc{Gillis}\\
                \emph{Assistant:}\\
                Maxime \textsc{Gobert}\\
            \end{flushleft}
        \end{minipage}
        \begin{minipage}[t]{0.45\textwidth}
            \begin{flushright} \large
                \emph{Auteurs:} \\
                Thomas \textsc{Bernard} \\
                Ugo \textsc{Proietti} \\
            \end{flushright}
        \end{minipage}\\[2ex]
        
        \vfill
        
        % Bottom of the page
        \begin{center}
            \begin{tabular}[t]{c c c}
                \includegraphics[height=1.5cm]{images/logoumons.jpg} &
                \hspace{0.3cm} &
            \end{tabular}
        \end{center}~\\
        
        {\large Année académique 2021-2022}
        
    \end{center}
\end{titlepage}

\tableofcontents

\newpage

\section{Description du problème}

\subfile{sections/description}

\section{Modélisation}

\subfile{sections/modelisation}

\section{Mise sous forme standard}

\subfile{sections/standard}

\section{Résolution du problème avec GNU OCtave}

\subfile{sections/resolution}

\end{document}
